% Преамбула документа
% Следующая команда определяет класс документа: статья (article)
\documentclass{article}
% Подключаем пакет inputenc и устанавливаем кодировку текста (utf8)
\usepackage[utf8]{inputenc}
                             % Здесь начинается преамбула
\usepackage[T2A]{fontenc}
\usepackage[russian]{babel}

\nonfrenchspacing            % Разрешаем увеличивать пробел
                             % после конца предложения
\usepackage[all]{xy}
\xyoption{frame}
% Эта команда означает начало основного текста
\begin{document}
%$\xymatrix@1@=0pt@M=0pt{A&B\\C&D}$
\begin{center}
$\xymatrix@R=1pt@C=1em{
  &                 	    & \txt{*} \ar[dl] \ar[dr]        \\
  & \txt{+} \ar[dl] \ar[dr] &                 		  &  5   \\
6 &                   		& 10
}$
\end{center}
Такая запись дерева эквивалентна следующей постфиксной записи:

\[ (\;(\;'nil'\;6\;'nil'\;)\;+\;(\;'nil'\;10\;'nil'\;)\;)\;*\;(\;'nil'\;5\;'nil'\;)\], где между каждыми двумя словами/скобками стоят пробелы




% Эта команда заканчивает документ, весь текст после нее игнорируется
\end{document}
*